\documentclass[12pt]{article}

% Packages
\usepackage{comment}
\usepackage{amsmath} % Advanced math typesetting
\usepackage{amsfonts}
\usepackage[utf8]{inputenc} % Unicode support (Umlauts etc.)
\usepackage{hyperref} % Add a link to your document
\usepackage[monochrome]{color} % Turning all of the colors stiched off

\begin{document}
\section{1}
\subsection{Definition}

\paragraph{Expected Profit}
optimal value of the objective function when considering all scenarios with equal probility. This means, first-stage decisions sufficiently robust such that in average the objective value is maximum.
e.g. the farm problem has an expected profit = 108,390

\paragraph{EVPI: Expected Value of Perfect Information}
Consider all possible scenarios, one knows in advance what scenario will happen on the following-stage. Hence, the best decisions will always be taken for all scenarios.
Then, do the average gain over all scenarios.
e.g. the farm problem has an EVPI = 115,406

\paragraph{Expected Value Solution}
Consider the mean situation, i.e. reducing all the uncertainty to one unique scenario, solving it optimally then applying the obtained decisions to each real scenario.
e.g. the farm problem has an expected value solution = 107,240

\paragraph{VSS: Value of the Stochastic Solution}
VSS = Expected Profit - Expected Value Solution = the loss of not considering uncertainty

\subsection{General Model Formulation}

\paragraph{Model}
\begin{align}
	\text{min } & c^{T} x + \mathrm{E}_{\xi} \mathrm{Q}(x,\xi)
\end{align}
s.t.:
\begin{align}
Ax &= b\\
x &\geq 0
\end{align}
where:
\begin{equation}
\mathrm{Q}(x,\xi) = \text{min } q^T y
\end{equation}
s.t.
\begin{align}
W y &= h - Tx\\
y &\geq 0
\end{align}


with the recourse or value function $\mathcal{Q}$:
\begin{equation}
\mathcal{Q}(x) = \mathrm{E}_{\xi} \mathrm{Q}(x,\xi)
\end{equation}
meaning it is the expectancy over all scenarios in $\xi$ for a given decision $x$.\\
\\
Depending on the case, computation of $\mathrm{Q}(x,\xi)$ can be separated:\\
\begin{equation}
\mathcal{Q}(x) = \mathrm{E}_{\xi} \mathrm{Q}(x,\xi) = \sum\limits_{i=1}^{n} \mathrm{E}_i \mathrm{Q}_i(x_i,\xi) = \sum\limits_{i=1}^{n} \mathcal{Q}_i(x_i)
\end{equation}

\end{document}